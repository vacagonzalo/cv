\section{Proyectos Seleccionados}

\newcommand{\pub}[5]{
	\parbox[t][][t]{\linewidth}{%
		\begin{small}
		\parbox{\linewidth}{{``#4''}}
		\smallbreak
		\parbox{\linewidth}{{#2}, {#1}}
		%\parbox{\linewidth}{\small{Keywords:}\footnotesize{ #5}}
		\parbox{\linewidth}{{\href{https://doi.org/#3}{#3}}}
		\end{small}
	}
	\bigbreak
	\smallskip
}

\pub{Oct. 2022}{Driver de Linux para esclavo AXI}{}{Un driver en espacio de kernel que administra las transacciones con un esclavo AXI sintetizado en lógica programable.}{
}

\pub{Ago. 2022}{Sistema Operativo de Tiempo Real \emph{myOS}}{}{Un sistema operativo de tiempo real para la arquitectura Cortex-M4 realizado desde cero. Implementa un scheduler round robin con prioridades, manejo de interrupciones y un sistema de stack estático para cada tarea.}{
}

\pub{Jun. 2022}{Evaluador de microcontroladores para misiones espaciales}{}{Un conjunto de herramientas que permite inyectar SEFI-SEU y obtener la figura de mérito de un microcontrolador. Proyecto impulsado por INVAP en el marco de la Maestría en Internet de las Cosas de la Universidad de Buenos Aires.}{
}

\pub{May. 2021}{Monitoreo Ambiental Integrado a Enterprise Buildings Integrator de Honeywell}{}{Un sistema flexible que une el protocolo MQTT para una red de sensores y los presenta en protocolo MODBUS al entorno propietario que se encuentra funcionando en planta, realizado en el marco de la Especialización en IoT de la Universidad de Buenos Aires}{
}

%\pub{Aug. 2018}{SRMPDS (Best paper)}{10.1145/3229710.3229759}{Flexible device sharing in PCIe clusters using
%Device~Lending}{%
%Paravirtualizaiton; KVM, Virtual Machines;
%}

