\section{Experiencia Laboral}

%Experiencia 1
\parbox[t][][t]{\linewidth}{
	\parbox{\linewidth}{
		\textbf{Embedded Software Engineer}
			\hfill
			{Feb. 2024 --- \phantom{Dec. 2099}}
		}
	\smallbreak
	%\smallskip
	\parbox{\linewidth}{Pump Control S.R.L.}
	%\smallskip
	\smallbreak
	%\bigskip
		\begin{itemize}
			\item{Programación en C/C++ de sistemas embebidos.}
			\item{Diseñé y desarrollé un Bootloader para la arquitectura ARM M0+ que protege la propiedad intelectual y permite actualizar el firmware en campo.}
			\item{Diseñé y desarrollé un gateway y sensores de terreno LoRa para la detección de gases de hidrocarburos.}
			\item{Modernicé el sistema de gestión y cifrado de la propiedad intelectual de la compañía.}
		\end{itemize}
	\smallbreak
    \emph{Pump Control} provee soluciones para el sector de oil and gas.
}

\bigskip
%Experiencia 2
\parbox[t][][t]{\linewidth}{
	\parbox{\linewidth}{
		\textbf{Ingeniero de Software Embebido}
			\hfill
			{Ene. 2022 --- Feb. 2024}
		}
	\smallbreak
	%\smallskip
	\parbox{\linewidth}{INVAP S.E.}
	%\smallskip
	\smallbreak
	%\bigskip
	\begin{itemize}
		\item Radar Primario Argentino:
		\begin{itemize}
	    	\item{Desarrollo en MPSoC de AMD/Xilinx.}
			\item{C++: Desarrollo de los servicios de recepción, configuración y sincronismo del radar.}
			\item{C y FreeRTOS: aplicaciones de tiempo real para los núcleos R5.}
			\item{Yocto y Petalinux:}
			\begin{itemize}
				\item{Implementé las recetas de Bitbake que permiten construir las imágenes de Petalinux.}
				\item{Pude resolver la falta de un \emph{Board Support Package (BSP)} para Petalinux en una placa diseñada por un proveedor.}
			\end{itemize}
			\item{Extenso uso de MQTT, Protobuffer, ZMQ, keepalived y OpenAMP.}
		\end{itemize}
		\item Satélite SABIA-Mar:
		\begin{itemize}
	    	\item{Diseñé e implemente en Python 3 la herramienta \emph{Sistema de Inyección de Soft-errors} (SISE). Este sistema simula los efectos lógicos de la radiación en un microcontrolador M4 y permite evaluar las técnicas de mitigación de errores.}
		\end{itemize}
	\end{itemize}
	\smallbreak
    \emph{INVAP} es la principal empresa tecnológica de Argentina (nuclear, aeroespacial y defensa).
}

%\bigskip
%Experiencia 3
%\parbox[t][][t]{\linewidth}{
%	\parbox{\linewidth}{
%		\textbf{Profesor Universitario}
%			\hfill
%			{Mar. 2022 --- \phantom{Dic. 2099}}
%		}
%	\smallbreak
%	%\smallskip
%	\parbox{\linewidth}{Universidad de Morón}
%	%\smallskip
%	\smallbreak
%	%\bigskip
%	Ingeniería Electrónica - Técnicas Digitales II.
%	\begin{itemize}
%	    \item{Modernicé la cátedra para incluir:}
%		\begin{itemize}
%			\item{Teoría y práctica de sistemas operativos de tiempo real.}
%			\item{Uso avanzado del lenguaje C.}
%			\item{Flujo profesional de trabajo: git, CMake, CMocka y Doxygen.}
%		\end{itemize}
%		\item{Implementé un laboratorio a partir del \emph{port} POSIX de FreeRTOS.}
%	\end{itemize}
%}

\bigskip
%Experiencia 4
\parbox[t][][t]{\linewidth}{
	\parbox{\linewidth}{
		\textbf{Ingeniero de Software Embebido}
			\hfill
			{Feb. 2021 --- Ene. 2022}
		}
	\smallbreak
	%\smallskip
	\parbox{\linewidth}{American Traffic S.A.}
	%\smallskip
	\smallbreak
	%\bigskip
	\begin{itemize}
	    \item{Implementé un sistema de telemetría para monitorear y controlar todos los radares de Vialidad Nacional.}
	    \item{Integré un sistema de perfiladores laser para el desarrollo del primer sistema \emph{Weigh in motion} (WIM) de la República Argentina.}
		\item{Implementé un programa que acondiciona una nube de puntos para su ingesta en la inteligencia artificial que clasifica los vehículos.}
	\end{itemize}
	\smallbreak
    \emph{ATSA} es un proveedor de soluciones de infraestructura de tránsito vehicular.
}

\bigskip
%%Experiencia 5
%\parbox[t][][t]{\linewidth}{
%	\parbox{\linewidth}{
%		\textbf{Auditor Externo}
%			\hfill
%			{Abr. 2018 --- Dic. 2020}
%		}
%	\smallbreak
%	%\smallskip
%	\parbox{\linewidth}{Picado, Levy De Angelis \& Asociados}
%	%\smallskip
%	\smallbreak
%	%\bigskip
%	\begin{itemize}
%	    \item{Auditorías de instalaciones.}
%	    \item{Asistencia a contratas.}
%	\end{itemize}
%	\smallbreak
%	Actividades realizadas al servicio de \emph{Telecentro SA}.
%
%    Realicé auditorías de tecnologías \emph{HFC} y \emph{GPON}.
%}

%%Experiencia 6
%\bigskip
%\parbox[t][][t]{\linewidth}{
%	{\parbox{\linewidth}{
%		\parbox{\linewidth}{
%			\textbf{Encargado de electrónica}
%			\hfill
%			{Abr. 2015 --- Abr. 2018}
%		}
%	}}
%	\smallbreak
%	\parbox{\linewidth}{Tecnotrans SRL}
%	\smallbreak
%	\begin{itemize}
%	    \item{Planificación y supervisión de la producción y servicio técnico.}
%	    \item{Compras técnicas.}
%	\end{itemize}
%	\smallbreak
%	\emph{Tecnotrans} es un fabricante de controladores de tránsito, semáforos y semáforos ferroviarios.
%}

%%Experiencia 7
%\bigskip
%\parbox[t][][t]{\linewidth}{
%	{\parbox{\linewidth}{
%		\parbox{\linewidth}{
%			\textbf{Instrumentista}
%			\hfill
%			{Ago. 2012 --- Abr. 2015}
%		}
%	}}
%	\smallbreak
%	\parbox{\linewidth}{Honeywell Process Solutions.}
%	\smallbreak
%	\begin{itemize}
%	    \item{Calibración de instrumental bajo norma ISO-IEC 17025.}
%	    \item{Mantenimiento de sistemas de detección de incendios.}
%	\end{itemize}
%	\smallbreak
%	Realicé viajes al interior del país para calibrar instrumental de varias industrias, entre los rubros más destacados, metalúrgica, cosmética, farmacéutica y petrolera.\\
%    \emph{Honeywell} es una empresa dedicada a la fabricación de sistemas industriales y aeroespaciales.
%}

